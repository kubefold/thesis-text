\chapter{Introduction}


\section{Preface}


\section{Motivation}

Predicting the conformational structure of proteins represents a fundamental challenge in biological sciences, with crucial significance in areas such as the pharmaceutical industry.
Understanding the three-dimensional structure of proteins is essential for designing new drugs.
Knowledge of the spatial structure of a protein allows us to understand its biological function, mechanism of action, and potential interactions with other molecules.
This is particularly important in the process of discovering new therapeutic substances.

The main motivation for creating the KubeFold project was recognizing a significant technological gap in the infrastructure for protein structure prediction.
Although algorithms such as AlphaFold ~\cite{alphafold3} have achieved unprecedented success in modeling protein conformations, their practical application often requires specialized knowledge in IT infrastructure, which creates a barrier for many researchers in the fields of molecular biology or bioinformatics.

Previous approaches to running AlphaFold were often based on manual configuration of the computational environment, which involved many challenges: from downloading and managing gigantic protein databases (on the order of terabytes), through configuring hardware resources (including specialized GPU accelerators), to optimizing workflow and managing computational artifacts.
This process was time-consuming, error-prone, and required competencies across multiple disciplines simultaneously.

KubeFold aims to overcome these challenges by providing an automated, scalable, and easy-to-use platform that hides the infrastructure layer from the end user.
By leveraging containerization technology and Kubernetes orchestration ~\cite{kubernetes, container_orchestration}, the system provides automated resource management and flexible scaling, allowing scientists to focus on biological research rather than technical aspects.


\section{Goals and Scope of Work}

The main goal of this work is to design and implement the KubeFold operator for the Kubernetes platform ~\cite{kubernetes}, which automates the process of running protein conformational structure prediction calculations using the AlphaFold algorithm ~\cite{alphafold3}.
The implementation of this goal includes the following tasks:

\begin{itemize}
    \item Developing the architecture of a Kubernetes operator ~\cite{k8s_operators}, which primarily:
    \subitem introduces the \textit{ProteinDatabase} resource abstracting the management of protein databases
    \subitem introduces the \textit{ProteinConformationPrediction} resource managing the actual protein conformation calculation
    \item Designing and implementing a component for managing protein databases, enabling automatic downloading and decompression of data files
    \item Preparing example infrastructure code (infrastructure as code), which enables running the operator in the Amazon Web Services (AWS) cloud with the connection of \textit{FSx for Lustre} volumes
    \item Creating documentation and installation instructions for the operator in a cluster.
\end{itemize}

The scope of work encompasses the full software development lifecycle: from analyzing problems in existing solutions, through architecture design, component implementation, to verification of system operation in a Kubernetes environment.

The project does not include modifications to the AlphaFold algorithm, but focuses on automating its execution in a cloud environment using containerization technology.
