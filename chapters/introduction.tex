\chapter{Wstęp}


\section{Wprowadzenie}


\section{Motywacja}

Przewidywanie struktury konformacyjnej białek stanowi fundamentalne wyzwanie w naukach biologicznych, mające kluczowe znaczenie między innymi w przemyśle farmaceutycznym.
Zrozumienie trójwymiarowej struktury białek jest niezbędne do projektowania nowych leków.
Znajomość struktury przestrzennej białka pozwala zrozumieć jego funkcję biologiczną, mechanizm działania oraz potencjalne interakcje z innymi cząsteczkami.
Jest to szczególnie istotne w procesie odkrywania nowych substancji terapeutycznych.

Główną motywacją do stworzenia projektu KubeFold było dostrzeżenie istotnej luki technologicznej w obszarze infrastruktury dla przewidywania struktur białkowych.
Choć algorytmy takie jak AlphaFold ~\cite{alphafold3} osiągnęły bezprecedensowy sukces w modelowaniu konformacji białek, ich praktyczne zastosowanie często wymaga specjalistycznej wiedzy z zakresu infrastruktury IT, co stanowi barierę dla wielu badaczy z dziedziny biologii molekularnej czy bioinformatyki.

Dotychczasowe podejścia do uruchamiania AlphaFold często opierały się na ręcznej konfiguracji środowiska obliczeniowego, co wiązało się z wieloma wyzwaniami: począwszy od pobierania i zarządzania gigantycznymi bazami danych białkowych (rzędu terabajtów), przez konfigurację zasobów sprzętowych (w tym wyspecjalizowanych akceleratorów GPU), aż po optymalizację przepływu pracy i zarządzanie artefaktami obliczeń.
Proces ten był czasochłonny, podatny na błędy i wymagał kompetencji z wielu dziedzin jednocześnie.

KubeFold ma na celu przezwyciężenie tych wyzwań poprzez dostarczenie zautomatyzowanej, skalowalnej i łatwej w użyciu platformy, która ukrywa warstwę infrastruktury przed użytkownikiem końcowym.
Dzięki wykorzystaniu technologii konteneryzacji i orkiestracji Kubernetes ~\cite{kubernetes, container_orchestration} system zapewnia zautomatyzowane zarządzanie zasobami i elastyczne skalowanie, umożliwiający naukowcom skupienie się na badaniach biologicznych, a nie na aspektach technicznych.


\section{Cele i zakres pracy}

Głównym celem pracy jest zaprojektowanie i implementacja operatora KubeFold dla platformy Kubernetes ~\cite{kubernetes}, który automatyzuje proces uruchamiania obliczeń przewidywania struktury konformacyjnej białek z wykorzystaniem algorytmu AlphaFold ~\cite{alphafold3}.
Realizacja tego celu obejmuje następujące zadania:

\begin{itemize}
    \item Opracowanie architektury operatora Kubernetes ~\cite{k8s_operators}, który przede wszystkim:
    \subitem wprowadza zasób \textit{ProteinDatabase} abstrahujący zarządzanie bazami danych białkowych
    \subitem wprowadza zasób \textit{ProteinConformationPrediction} zarządzający faktycznym obliczeniem konformacji białka
    \item Zaprojektowanie i implementacja komponentu do zarządzania bazami danych białkowych, umożliwiającego automatyczne pobieranie i dekompresję plików z danymi
    \item Przygotowanie przykładowego kodu infrastruktury (z ang. \textit{infrastructure as code}), który umożliwia uruchomienie operatora w chmurze Amazon Web Services (AWS) wraz z podpięciem wolumenów \textit{FSx for Lustre}
    \item Wykonanie dokumentacji i instrukcji instalacji operatora w klastrze.
\end{itemize}

Zakres pracy obejmuje pełny cykl wytwarzania oprogramowania: od analizy problemów w istniejących rozwiązaniach, przez projekt architektury, implementację komponentów, aż po weryfikację działania systemu w środowisku Kubernetes.

Projekt nie obejmuje modyfikacji algorytmu AlphaFold, a skupia się na automatyzacji jego uruchamiania w środowisku chmurowym z wykorzystaniem technologii konteneryzacji.
