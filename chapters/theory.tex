\chapter{Zagadnienie teoretyczne}


\section{Problem przewidywania konformacji białek}
Przewidywanie konformacji białek to kluczowy problem biologii obliczeniowej.
Polega na określeniu trójwymiarowej struktury białka na podstawie sekwencji aminokwasów.
Struktura przestrzenna białka bezpośrednio wpływa na jego funkcję biologiczną.
Dokładne przewidywanie konformacji pozwala zrozumieć mechanizmy działania białek oraz projektować nowe leki.
Jest to zadanie trudne ze względu na ogromną liczbę możliwych konformacji oraz złożone oddziaływania między atomami.
Problem ten staje się szczególnie istotny dla białek, których struktury nie udało się wyznaczyć metodami eksperymentalnymi ze względu na trudności techniczne.
Zastosowanie przewidywanych struktur w medycynie spersonalizowanej oraz farmacji stanowi jeden z najbardziej obiecujących kierunków rozwoju współczesnej biotechnologii.

\subsection{Kodowanie trójwymiarowej struktury białka}
Aminokwasy w łańcuchu polipeptydowym kodują strukturę przestrzenną białka poprzez swoją sekwencję i właściwości fizykochemiczne.
Każdy aminokwas posiada specyficzne cechy, takie jak hydrofobowość, ładunek elektryczny czy wielkość, które determinują możliwe oddziaływania i ułożenie w przestrzeni.
Sposób ułożenia aminokwasów w przestrzeni jest zapisany w tak zwanym kodzie konformacyjnym, który opisuje różne poziomy organizacji struktury białka.

Strukturę przestrzenną białka opisuje się na kilku hierarchicznych poziomach organizacji:

\begin{itemize}
    \item Struktura pierwszorzędowa - liniowa sekwencja aminokwasów połączonych wiązaniami peptydowymi, która zawiera pełną informację niezbędną do uformowania struktury natywnej
    \item Struktura drugorzędowa - lokalne motywy strukturalne powstające dzięki wiązaniom wodorowym między grupami peptydowymi, takie jak helisy alfa i harmonijki beta
    \item Struktura trzeciorzędowa - globalne ułożenie całego łańcucha polipeptydowego stabilizowane różnymi oddziaływaniami między aminokwasami
    \item Struktura czwartorzędowa - sposób asocjacji kilku łańcuchów polipeptydowych w funkcjonalny kompleks białkowy
\end{itemize}

\subsection{Krystalografia rentgenowska}
Krystalografia rentgenowska to podstawowa metoda doświadczalna określania struktury białek.
Polega na analizie dyfrakcji promieni rentgenowskich na krysztale białka.
Promienie rentgenowskie rozpraszają się na elektronach atomów tworzących białko.
Wzór dyfrakcyjny jest rejestrowany i analizowany matematycznie.
Na tej podstawie tworzy się mapa gęstości elektronowej.
Analiza danych dyfrakcyjnych wymaga zastosowania złożonych algorytmów matematycznych i wiedzy z zakresu krystalografii.
Interpretacja map gęstości elektronowej stanowi wyzwanie wymagające często ręcznego dopasowania modelu białka przez doświadczonych badaczy.

Proces krystalizacji białka jest często najtrudniejszym etapem, wymagającym znacznych nakładów finansowych i zaawansowanej infrastruktury laboratoryjnej.

\subsection{Komputerowe metody obliczania konformacji białka}
Metody komputerowe przewidywania struktury białek dzielą się na kilka głównych kategorii.
Metody oparte na homologii wykorzystują podobieństwo sekwencji do białek o znanej strukturze.
Metody ab initio próbują przewidywać strukturę wyłącznie na podstawie sekwencji i praw fizyki.
Metody fold recognition (rozpoznawania fałdowania) dopasowują sekwencję do znanych motywów strukturalnych.
Każda z tych metod ma swoje ograniczenia i najlepiej sprawdza się w określonych warunkach, zależnie od dostępności danych referencyjnych.
Hybrydy tych metod często dają lepsze wyniki niż pojedyncze podejścia, łącząc zalety różnych technik przewidywania.

Tradycyjne podejścia obliczeniowe obejmują symulacje dynamiki molekularnej i metody Monte Carlo.
Wykorzystują one pola siłowe opisujące oddziaływania między atomami.
Obliczenia te są bardzo kosztowne obliczeniowo i często ograniczone do małych białek.
W ostatnich latach metody uczenia maszynowego i głębokiego odniosły znaczący sukces w tym obszarze.
Postęp w sprzęcie obliczeniowym, szczególnie w obszarze akceleratorów GPU i specjalizowanych procesorów tensorowych, znacząco przyspieszył obliczenia struktur białkowych.
Rozwój infrastruktury chmurowej umożliwił demokratyzację dostępu do zaawansowanych narzędzi bioinformatycznych.

Przełomem w dziedzinie było opracowanie algorytmu AlphaFold przez firmę Google DeepMind.
Algorytm ten wykorzystuje sieci neuronowe do przewidywania struktury białek z dokładnością porównywalną do metod eksperymentalnych.
AlphaFold2 zrewolucjonizował dziedzinę, wygrywając konkurs CASP14 (Critical Assessment of protein Structure Prediction) w 2020 roku.
Metody oparte na uczeniu głębokim znacząco przyspieszyły tempo odkryć naukowych w biologii strukturalnej.
Publiczne udostępnienie modelu AlphaFold umożliwiło badaczom na całym świecie wykorzystanie tej technologii w projektach badawczych i medycznych.
Integracja wyników AlphaFold z innymi technikami bioinformatycznymi otworzyła nowe możliwości w analizie interakcji białko-białko i projektowaniu leków.
Najnowsza wersja algorytmu-AlphaFold 3 - osiąga dokładność przewidywania struktury białek na poziomie 90--95\% w przypadku pojedynczych łańcuchów polipeptydowych, a dla kompleksów białkowych dokładność sięga 80--85\%, co stanowi znaczący postęp w porównaniu do poprzednich wersji.


\section{Algorytm AlphaFold firmy Google Deepmind}

\subsection{Informacje wstępne}

\subsection{Implementacja algorytmu}

\subsection{Omówienie działania programu}

\subsection{Wymagane zasoby obliczeniowe}


\section{Obliczenia w klastrze obliczeniowym}

\subsection{Platforma Kubernetes}

\subsection{Architektura i komponenty klastra Kubernetes}

\subsection{Dostępne zasoby w klastrze obliczeniowym}

\subsubsection{Pod}

\subsubsection{Job}

\subsubsection{PersistentVolume}

\subsubsection{PersistentVolumeClaim}

\subsubsection{StorageClass}

\subsection{Klastry Kubernetes w chmurze}

\subsection{Zarządzana usługa Kubernetes w Amazon Web Services}

\subsection{Wolumeny Lustre}

\subsection{Magazyn obiektów Amazon S3}

\subsection{Usługa AWS End User Messaging}


\section{Koncepcja operatora klastra}

\subsection{Architektura operatora klastra}

\subsection{Język programowania Go}
