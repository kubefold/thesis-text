\chapter{Realizacja projektu}
Niniejszy rozdział omówi architekturę i sposób implementacji projektu KubeFold.


\section{Koncepcja platformy KubeFold}
KubeFold to oprogramowanie, które działa na platformie Kubernetes w trybie Operatora.
Projekt składa się z następujących komponentów:
\begin{itemize}
    \item Operator Kubernetes (z ang. \textit{operator}) - komponent, który zarządzania logiką biznesową.
    Dba o poprawne uruchamianie i zarządzanie podami służącymi do pobierania baz danych białek oraz obliczeń.
    Komponent został dokładnie opisany w sekcji~\ref{subsec:component-operator}.
    \item Komponent pobierania (z ang. \textit{downloader}) - służy do kontroli procesu pobierania i dekompresji baz danych białek.
    Dokładny opis znajduje się w sekcji~\ref{subsec:component-downloader}.
    \item Komponent zarządzający obliczeniami (z ang. \textit{manager}) - obsługuje dodatkowe działania wokół obliczeń takie jak wysyłka artefaktów bądź wysyłanie powiadomienia.
    Opisany został w sekcji~\ref{subsec:component-manager}.
\end{itemize}


\section{Użyte technologie}
Wszystkie trzy komponenty zostały napisane w języku Go w wersji 1.24 (więcej o nim w sekcji~\ref{subsec:go-programming-language}).
Był to naturalny wybór, ponieważ Go jest de facto standardem w przypadku tworzenia oprogramowania chmurowego.
Kod źródłowy komponentów był archiwizowany w repozytorium Git~\cite{git} w serwisie GitHub~\cite{github}.

W celu kompilacji aplikacji i spakowania plików wykonywalnych do obrazu kontenera użyto standardowego buildera od Dockera.

Dystrybucja obrazów odbywa się za pomocą rejestru Github Container Registry (\textit{ghcr.io})~\cite{ghcr}.
Obrazy są automatycznie budowane przy każdym wypchnięciu tagu Git przez narzędzie GitHub Actions~\cite{github_actions}.
Flow działania zostało przestawione na rys.~\ref{fig:docker-images-flow}.

\begin{figure}[htbp]
    \centering
    \includegraphics[width=0.8\textwidth]{images/images}
    \caption{Development workflow between GitHub and cluster}
    \label{fig:docker-images-flow}
\end{figure}

Komponent Operatora (szerzej opisany w~\ref{subsec:component-operator}) został utworzony na podstawie framework KubeBuilder~\cite{kubebuilder}.
KubeBuilder to framework, który służy do pisania natywnych operatorów Kubernetes.
Tworzy on strukturę katalogów projektu oraz od razu zapewnia zarządzanie elementami takimi jak:
\begin{itemize}
    \item Elekcja lidera operatora-niektóre instancje operatorów mogą być uruchomione na wielu węzłach klastra jednocześnie, aczkolwiek jeżeli istnieje potrzeba wybrania jednej wyróżnionej instancji, to elekcja lidera wybiera go.
    \item Generowanie skryptów instalacyjnych-KubeBuilder ma możliwość wygenerowania gotowych skryptów instalacyjnych, które mogą później posłużyć do automatycznej instalacji operatora i wszystkich jego zależności w klastrze jednym wykonaniem polecenia \textit{kubectl apply}.
    \item Definicje ról i permisji.
    Kubernetes korzysta z modelu autoryzacji o nazwie \textit{RBAC} (Role-Based Access Control)~\cite{k8s_rbac}.
    KubeBuilder wykrywa jakie permisje powinna mieć instancjadd a operatora i upewnia się, że odpowiednie permisje zostały mu przyznane.
\end{itemize}


\section{Architektura rozwiązania}


\section{Opis komponentów}

\subsection{Komponent operatora}\label{subsec:component-operator}

\subsection{Komponent pobierania}\label{subsec:component-downloader}

\subsection{Komponent zarządzający obliczeniami}\label{subsec:component-manager}