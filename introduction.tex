%! suppress = MissingLabel


\chapter{Introduction}

\section{Preface}

The convergence of computational biology and cloud computing has revolutionized how scientific research is conducted in the field of molecular biology. Protein structure prediction, once a challenging task requiring specialized hardware and expertise, has been transformed by breakthroughs in artificial intelligence algorithms such as AlphaFold. However, the practical application of these advanced algorithms remains hindered by complex infrastructure requirements and technical barriers.

This thesis introduces KubeFold, a Kubernetes-based platform that bridges the gap between cutting-edge AI algorithms for protein structure prediction and their practical application in research environments. By leveraging container orchestration technology and cloud-native principles, KubeFold automates the deployment, scaling, and management of computational resources needed for protein structure prediction, making advanced bioinformatics tools more accessible to researchers without specialized IT knowledge.

The work presented here demonstrates how modern software engineering practices, particularly the Kubernetes Operator pattern, can be applied to scientific computing workflows to create robust, scalable, and user-friendly systems. The developed solution abstracts away the complexities of infrastructure management, database handling, and computational resource allocation, allowing scientists to focus on their research rather than technical implementation details.

Importantly, this thesis has an engineering and systems focus: its aim is not to improve the accuracy of protein structure predictions or to propose new bioinformatics algorithms, but to provide and simplify the infrastructure required to run such advanced models reliably and at scale.

As computational demands in life sciences continue to grow, platforms like KubeFold represent an important step toward democratizing access to advanced computational tools and accelerating scientific discovery in structural biology and related fields.


\section{Motivation}

Predicting the conformational structure of proteins represents a fundamental challenge in biological sciences, with crucial significance in areas such as the pharmaceutical industry.
Understanding the three--dimensional structure of proteins is essential for designing new drugs.
Knowledge of the protein spatial structure allows us to understand its biological function, mechanism of action, and potential interactions with other molecules.
This is particularly important in the process of discovering new therapeutic substances.

The main motivation for creating the KubeFold project was recognizing a significant technological gap in the infrastructure for protein structure prediction.
Although algorithms such as AlphaFold ~\cite{alphafold3} have achieved unprecedented success in modeling protein conformations, their practical application often requires specialized knowledge in IT infrastructure, which creates a barrier for many researchers in the fields of molecular biology or bioinformatics.

Previous approaches to running AlphaFold were often based on manual configuration of the computational environment, which involved many challenges: from downloading and managing gigantic protein databases (on the order of terabytes), through configuring hardware resources (including specialized GPU accelerators), to optimizing workflow and managing computational artifacts.
This process was time--consuming, error--prone, and required competencies across multiple disciplines simultaneously.

KubeFold aims to overcome these challenges by providing an automated, scalable, and easy--to--use platform that hides the infrastructure layer from the end user.
By leveraging containerization technology and Kubernetes orchestration ~\cite{kubernetes, container_orchestration}, the system provides automated resource management and flexible scaling, allowing scientists to focus on biological research rather than technical aspects.


\section{Goals and scope of work}

The main goal of this work is to design and implement the KubeFold operator for the Kubernetes platform ~\cite{kubernetes}, which automates the process of running protein conformational structure prediction calculations using the AlphaFold algorithm ~\cite{alphafold3}.
The implementation of this goal includes the following tasks:

\begin{itemize}
    \item Developing the architecture of a Kubernetes operator ~\cite{k8s_operators}, which primarily:
    \subitem introduces the \textit{ProteinDatabase} resource abstracting the management of protein databases,
    \subitem introduces the \textit{ProteinConformationPrediction} resource managing the actual protein conformation calculation,
    \item Designing and implementing a component for managing protein databases, enabling automatic downloading and decompression of data files,
    \item Preparing example infrastructure code (infrastructure as code), which enables running the operator in the Amazon Web Services (AWS) cloud with the connection of \textit{FSx for Lustre} volumes,
    \item Creating documentation and installation instructions for the operator in a cluster.
\end{itemize}

KubeFold should serve large organizations and institutions requiring enterprise-grade automation and cloud-native solutions for protein structure prediction. Unlike existing solutions such as Foldy that target individual researchers through web interfaces, KubeFold addresses the needs of bioinformatics centers, pharmaceutical companies, and research institutions that process multiple protein predictions simultaneously and require complete lifecycle automation. KubeFold should provide declarative infrastructure management and automated resource scaling, making it particularly suitable for environments where computational demands vary significantly and operational overhead must be minimized.

The scope of work encompasses the full software development lifecycle: from analyzing problems in existing solutions, through architecture design, component implementation, to verification of system operation in a Kubernetes environment.

The project does not include modifications to the AlphaFold algorithm and does not aim to improve prediction quality; it focuses exclusively on automating its execution in a cloud environment using containerization technology and Kubernetes orchestration.
