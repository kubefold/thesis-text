\chapter{Results}


\section{Implementation of the KubeFold Project}

W ramach projektu z sukcesem udało się:
\begin{itemize}
    \item Zaprojektowano i zaimplementowano architekturę operatora \textit{KubeFold} dla platformy Kubernetes.
    \item Utworzono dwa niestandardowe zasoby (CRD):
    \begin{itemize}
        \item \textit{ProteinDatabase} – zarządzanie bazami danych białek,
        \item \textit{ProteinConformationPrediction} – uruchamianie obliczeń konformacyjnych.
    \end{itemize}
    \item Opracowano komponent do automatycznego pobierania i dekompresji baz danych białek.
    \item Przygotowano infrastrukturę jako kod (IaC) umożliwiającą uruchomienie operatora w chmurze AWS (EKS + FSx for Lustre).
    \item Stworzono dokumentację oraz instrukcje instalacji operatora w klastrze Kubernetes.
    \item Przeprowadzono kilka obliczeń konformacji białek w celu przetestowania rozwiązania.
\end{itemize}


\section{Porównanie innych metod uruchamiania predykcji konformacji białek}

Algorytm AlphaFold znacząco wpłynął na rozwój bioinformatyki, oferując przełomowe możliwości przewidywania struktury przestrzennej białek. Mimo to, jego praktyczne wykorzystanie wiąże się z istotnymi wymaganiami obliczeniowymi i technologicznymi. W odpowiedzi na te wyzwania powstały alternatywne rozwiązania, takie jak \textbf{OpenFold} oraz \textbf{ColabFold}, których celem było uproszczenie procesu predykcji i zwiększenie jego dostępności. Na tle tych narzędzi wyróżnia się jednak \textbf{KubeFold} – opracowany w ramach niniejszej pracy system, który zapewnia znacznie wyższy poziom automatyzacji, skalowalności oraz integracji z infrastrukturą chmurową.

\subsection*{OpenFold}

OpenFold to otwartoźródłowa reimplementacja AlphaFold2, napisana w języku Python z wykorzystaniem biblioteki PyTorch. Projekt ten adresowany jest przede wszystkim do zespołów badawczych i instytucji, które dysponują rozbudowaną infrastrukturą obliczeniową oraz potrzebą pełnej kontroli nad procesem predykcji. OpenFold umożliwia m.in. trenowanie modeli od podstaw, testowanie modyfikacji architektury oraz uruchamianie zadań w środowiskach rozproszonych (HPC, chmura).

Z uwagi na zaawansowany charakter narzędzia, jego wdrożenie wymaga wysokich kompetencji z zakresu uczenia maszynowego, zarządzania kontenerami oraz konfiguracji środowisk chmurowych. Konieczne są również znaczące zasoby sprzętowe (np. karty GPU klasy A100), co może stanowić barierę wejścia dla mniejszych zespołów.

\subsection*{ColabFold}

ColabFold to uproszczona wersja AlphaFold2, dostępna poprzez interaktywne notebooki uruchamiane w środowisku Google Colaboratory. Dzięki integracji z narzędziem MMseqs2, proces dopasowywania sekwencji przebiega znacznie szybciej, a użytkownik nie musi samodzielnie konfigurować środowiska ani instalować oprogramowania.

ColabFold skierowany jest przede wszystkim do osób indywidualnych oraz zespołów, które nie dysponują zaawansowaną infrastrukturą obliczeniową. Mimo niskiego progu wejścia, rozwiązanie to posiada liczne ograniczenia: nie umożliwia trenowania modeli, jest zależne od ograniczeń platformy Colab (czas sesji, dostępność GPU), a jego zastosowanie ogranicza się głównie do prostych, jednostkowych zadań predykcyjnych.

\subsection*{Zalety podejścia KubeFold}

Opracowana w ramach niniejszej pracy platforma \textbf{KubeFold} stanowi nowoczesną i elastyczną alternatywę dla powyższych rozwiązań. Dzięki wykorzystaniu technologii Kubernetes oraz integracji z usługami Amazon Web Services (takimi jak FSx for Lustre czy S3), KubeFold automatyzuje cały proces uruchamiania predykcji konformacji białek – od pobierania baz danych, przez uruchamianie obliczeń, aż po archiwizację wyników i wysyłanie powiadomień.

W przeciwieństwie do OpenFold, KubeFold nie wymaga ręcznej konfiguracji zadań ani zarządzania infrastrukturą – wszystkie komponenty uruchamiane są automatycznie na podstawie zasobów YAML, co znacząco upraszcza proces.
W porównaniu z ColabFold, platforma KubeFold oferuje znacznie większą skalowalność i kontrolę nad zasobami obliczeniowymi – pozwalając na jednoczesne uruchamianie wielu zadań oraz przydzielanie im odpowiednich zasobów (CPU, GPU, pamięć, przestrzeń dyskowa).

Podejście to łączy zalety elastyczności i mocy obliczeniowej z łatwością użycia, co czyni KubeFold rozwiązaniem bardziej kompleksowym i przystosowanym do rzeczywistych potrzeb środowisk badawczych oraz zastosowań produkcyjnych.

\begin{table}[H]
    \centering
    \begin{tabular}{|p{4cm}|p{3.7cm}|p{3.7cm}|p{3.7cm}|}
        \hline
        \textbf{Cechy}                              & \textbf{OpenFold}                      & \textbf{ColabFold}                           & \textbf{KubeFold (proponowane)}                       \\
        \hline
        \textbf{Platforma}                          & Klastery HPC, chmura (AWS, GCP)        & Google Colab, środowiska lokalne & Kubernetes + AWS EKS \\
        \hline
        \textbf{Wymagania sprzętowe}                & Wysokie (GPU A100, RAM 64GB+)          & Niskie (darmowe GPU w Colab) & Średnie-wysokie, dynamicznie skalowalne \\
        \hline
        \textbf{Automatyzacja}                      & Ręczna konfiguracja zadań i środowiska & Brak, obsługa ręczna przez użytkownika & Pełna automatyzacja przez operatora Kubernetes \\
        \hline
        \textbf{Skalowalność}                       & Wysoka, ale trudna w konfiguracji      & Ograniczona do pojedynczych predykcji & Bardzo wysoka, wspiera przetwarzanie równoległe \\
        \hline
        \textbf{Łatwość wdrożenia}                  & Wysoki próg wejścia (DevOps, ML)       & Bardzo łatwa, odpowiednia dla początkujących & Średni próg, wymaga podstawowej znajomości Kubernetes \\
        \hline
        \textbf{Integracja z usługami zewnętrznymi} & Wymaga implementacji własnej           & Brak                                         & Tak – AWS S3, FSx, SNS, IAM                           \\
        \hline
    \end{tabular}
    \caption{Porównanie infrastruktury obliczeniowej: OpenFold, ColabFold i KubeFold}
\end{table}

\section{Challenges}


\section{Conclusions}