%! suppress = MissingLabel
\chapter{Results}
%
%\section{Implementation of the KubeFold Project}
%
%Within the project, the following goals were successfully accomplished:
%\begin{itemize}
%    \item Designed and implemented the \textit{KubeFold} operator for the Kubernetes platform.
%    \item Created two custom resources (CRDs):
%    \begin{itemize}
%        \item \textit{ProteinDatabase} – managing protein databases,
%        \item \textit{ProteinConformationPrediction} – running protein conformation prediction computations.
%    \end{itemize}
%    \item Developed a component for automatic downloading and decompression of protein databases.
%    \item Prepared infrastructure as code (IaC) allowing for quick operator deployment in AWS cloud (EKS + FSx for Lustre).
%    \item Created documentation and installation instructions for the Kubernetes operator.
%    \item Conducted several protein conformation computations to test the solution.
%\end{itemize}

%This is correct english statement
\section{Comparison of alternative solutions}

The AlphaFold algorithm has significantly influenced the development of bioinformatics, offering breakthrough capabilities in predicting protein spatial structures.
However, its practical use comes with significant computational and technological requirements.
In response to these challenges, alternative solutions have emerged, such as \textbf{OpenFold}\cite{openfold} and \textbf{ColabFold}\cite{colabfold}, which aimed to simplify the prediction process and increase its accessibility.
However, \textbf{KubeFold} – the system developed as part of this work – stands out among these tools, providing a much higher level of automation, scalability, and cloud infrastructure integration.

\subsection*{OpenFold}

OpenFold is an open--source reimplementation of AlphaFold2, written in Python using the PyTorch library.
This project is primarily targeted at research teams and institutions that have extensive computational infrastructure and need full control over the prediction process.
OpenFold enables, among other things, training models from scratch, testing architecture modifications, and running tasks in distributed environments (HPC, cloud).

Due to the advanced nature of the tool, its implementation requires high expertise in machine learning, container management, and cloud environment configuration.
Significant hardware resources (e.g., A100--class GPU cards) are also necessary, which may be an entry barrier for smaller teams.

\subsection*{ColabFold}

ColabFold is a simplified version of AlphaFold2, available through interactive notebooks running in the Google Colaboratory environment.
Thanks to integration with the MMseqs2 tool, the sequence alignment process is much faster, and users don't need to configure the environment or install software themselves.

ColabFold is primarily aimed at individuals and teams that don't have advanced computational infrastructure.
Despite the low entry threshold, this solution has numerous limitations: it doesn't allow model training, is dependent on Colab platform limitations (session time, GPU availability), and its application is mainly limited to simple, single predictive tasks.

\subsection*{Advantages of the KubeFold Approach}

The \textbf{KubeFold} platform developed as part of this work represents a modern and flexible alternative to the above solutions.
By utilizing Kubernetes technology and integration with Amazon Web Services (such as FSx for Lustre and S3), KubeFold automates the entire process of running protein conformation predictions – from downloading databases, through running calculations, to archiving results and sending notifications.

Unlike OpenFold, KubeFold doesn't require manual task configuration or infrastructure management – all components are automatically launched based on YAML resources, which significantly simplifies the process.
Compared to ColabFold, the KubeFold platform offers much greater scalability and control over computational resources – allowing for simultaneous running of multiple tasks and allocating appropriate resources (CPU, GPU, memory, disk space).
%TODO(matisiekpl): I don't know if this is enough
This scalability is achieved due to being built on top of Kubernetes platform.

This approach combines the advantages of flexibility and computational power with ease of use, making KubeFold a more comprehensive solution adapted to the real needs of research environments and production applications.

A detailed comparison of the computational infrastructure for OpenFold, ColabFold, and KubeFold is presented in Table~\ref{tab:comparison}.

\begin{table}[H]
    \centering
    \small
    \begin{tabularx}{\textwidth}{|X|X|X|X|}
        \hline
        \textbf{Features}                           & \textbf{OpenFold}                         & \textbf{ColabFold}                & \textbf{KubeFold (proposed)}                          \\
        \hline
        \textbf{Platform}                           & HPC clusters, cloud (AWS, GCP)            & Google Colab, local environments & Kubernetes + AWS EKS \\
        \hline
        \textbf{Hardware requirements}              & High (GPU A100, RAM 64GB+)                & Low (free GPU in Colab) & Medium--high, dynamically scalable \\
        \hline
        \textbf{Automation}                         & Manual task and environment configuration & None, manual handling by user & Full automation through Kubernetes operator \\
        \hline
        \textbf{Scalability}                        & High, but difficult to configure          & Limited to single predictions & Very high, supports parallel processing \\
        \hline
        \textbf{Ease of deployment}                 & High entry threshold (DevOps, ML)         & Very easy, suitable for beginners & Medium threshold, requires basic Kubernetes knowledge \\
        \hline
        \textbf{Integration with external services} & Requires own implementation               & None                              & Yes – AWS S3, FSx, SNS, IAM                           \\
        \hline
    \end{tabularx}
    \caption{Comparison of computational infrastructure: OpenFold, ColabFold, and KubeFold}
    \label{tab:comparison}
\end{table}

\section{Challenges}

The main challenge was developing an appropriate architecture for the KubeFold platform that can operate on multiple cluster nodes simultaneously, optimizing costs and computation speed.
By dividing the download process into multiple pods, it was possible to increase the potential throughput of protein database downloads.
Additionally, separating the \textit{Aligning} and \textit{Predicting} phases allowed for optimal utilization of CPU and GPU computational resources.
This setup enables cost savings in the public cloud.

The second challenge was the proper implementation of the \textit{downloader} component.
This component must simultaneously download and decompress data.
If this process were to occur sequentially, the temporarily used disk space would be much larger than the actual size of the database.

The third challenge was setting up appropriate infrastructure for testing the KubeFold platform.
The project runs using multiple AWS cloud services, which requires thoughtful integration of each component.
Setting up appropriate subnet configurations, VPC, and CSI drivers required a significant amount of time.

\section{Conclusions}

The KubeFold project successfully addresses the challenges of protein structure prediction by leveraging modern cloud technologies and Kubernetes orchestration.
The implementation demonstrates that complex scientific computations can be effectively managed through a declarative, operator--based approach, significantly reducing operational overhead.
The platform's architecture, combining AWS services like FSx for Lustre and S3 with Kubernetes orchestration, provides a scalable and cost--effective solution for running AlphaFold predictions.
The separation of database management and computation phases, along with automated resource provisioning and cleanup, makes the system particularly suitable for research environments requiring high--throughput protein structure predictions.
This work demonstrates how cloud--native technologies can transform complex scientific workflows into manageable, automated processes, potentially accelerating discoveries in structural biology and drug development.